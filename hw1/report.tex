\documentclass[12pt,a4paper,oneside]{article}
\usepackage{amsmath,amsthm}
\usepackage{}
\usepackage{ctex}

\title{计算方法编程作业一实验报告}
\author{张博厚 PB22071354}
\date{}

\begin{document}
\maketitle

\section{实验内容}
本实验要求编程求解一反射问题的二维简化模型: 将镜面假定为圆形, 给定观察点P与物点Q, 
输出反射点T和像点R.

\section{算法}
\subsection{计算反射点T}
对于反射点T,参考《Computational Mirror Cup and Saucer Art》中的方法,使用二分法数值求解:
假定观察点P在x轴负半轴,物点Q在第二象限, P与Q均在圆外, G为过点P在第二象限内所作切线的
切点, 如图1所示.
\end{document}