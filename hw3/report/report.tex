\documentclass[12pt,a4paper,oneside]{article}
\usepackage{amsmath,amsthm}
\usepackage{graphicx}
\usepackage{ctex}
\usepackage{float}
\usepackage{subfigure,subcaption}
\usepackage{caption}
\usepackage{geometry}

\geometry{a4paper,scale=0.8}

\title{计算方法编程作业3实验报告}
\author{张博厚 PB22071354}
\date{}

\begin{document}
\maketitle
\tableofcontents
\newpage

\section{实验目的}
使用C++实现带规范方法的反幂法, 求解给定矩阵的按模最小特征值及相应的特征向量, 其中
迭代方程$X^{k+1} = A^{-1}Y^k$使用LU分解(Doolittle分解)来求解。

\section{算法}
按教材所述反幂法与Doolittle分解算法来执行。
\section{实验结果}\noindent
实验结果如下:\\
A1:
\begin{table}[h!]
    \centering
    \resizebox{\textwidth}{30mm}{
    \begin{tabular}{| c |c | c | c|}
        \hline
       k &  $X^k$ & $Y^k$ & $\lambda$\\
        \hline
        0 & (1, 1, 1, 1, 1) & (1, 1, 1, 1, 1) & 1\\
        \hline
        1 & (630, -1120, 630, -120, 5) & (0.5625, -1, 0.5625, -0.1071, 0.00446) & 1120\\
        \hline
        2 & (146253, -297849, 196175, -45114.4, 2377.25) & (0.49103, -1, 0.658639, -0.151467, 0.00798141) & 297849\\
        \hline
        3 & (149113, -304047, 200595, -46244.7, 2446.56) & (0.490426, -1, 0.65975, -0.152097, 0.00804664) & 304047\\
        \hline
        4 & (149157, -304142, 200661, -46261.1, 2447.54) & (0.49042, -1, 0.659762, -0.152104, 0.00804735) & 304142\\
        \hline
        5 & (149158, -304143, 200662, -46261.3, 2447.55) & (0.49042, -1, 0.659762, -0.152104, 0.00804736) & 304143\\
        \hline
        6 & (149158, -304143, 200662, -46261.3, 2447.55) & (0.49042, -1, 0.659762, -0.152104, 0.00804736) & 304143\\
        \hline
        7 & (149158, -304143, 200662, -46261.3, 2447.55) & (0.49042, -1, 0.659762, -0.152104, 0.00804736) & 304143\\
        \hline
    \end{tabular}}
\end{table}\par
故A1的按模最小的特征值为$1/304143 = 3.2879\times 10^{-6}$, 对应的特征向量为(0.49042, -1, 0.659762, -0.152104, 0.00804736).\\
A2:
\begin{table}[h!]
    \centering
    \resizebox{\textwidth}{28mm}{
    \begin{tabular}{| c |c | c | c|}
        \hline
       k &  $X^k$ & $Y^k$ & $\lambda$\\
        \hline
        0 & (1,1,1,1) & (1,1,1,1) & 1\\
        \hline
        1 & (0, 2, -0, 1) & (0, 1, -0, 0.5) & 2\\
        \hline
        2 & (-0.625, 5.625, -2.375, 3.5) & (-0.111111, 1, -0.422222, 0.622222) & 5.625\\
        \hline
        3 & (-0.933333, 8.07778, -3.43333, 5.04444) & (-0.115543, 1, -0.425034, 0.624484) & 8.07778\\
        \hline
        4 & (-0.93621, 8.08992, -3.44378, 5.05433) & (-0.115725, 1, -0.425687, 0.624769) & 8.08992\\
        \hline
        5 & ( -0.936712, 8.09382, -3.44549, 5.05681) & ( -0.115732, 1, -0.425694, 0.624774) & 8.09382\\
        \hline
        6 & (-0.936719, 8.09386, -3.44551, 5.05684) & (-0.115732, 1, -0.425695, 0.624775) & 8.09386\\
        \hline
    \end{tabular}}
\end{table}\par
故A2的按模最小的特征值为$1/8.09386 = 0.12355$, 对应的特征向量为(-0.115732, 1, -0.425695, 0.624775).

\section{结果分析}\noindent
1. 是否有"A的按模最小特征值越趋近0, 收敛越快"?\par
比较可知, A1的按模最小特征值更趋近0, 但迭代次数多于A2, 因此不满足上述条件。 事实上, 反幂法
收敛快慢取决于A的按模最小特征值与按模次小特征值的比值大小,该比值越接近0, 则收敛越快。\\
2.估计每次特征值中是否遇到问题?如何解决?\par
通过观察每次迭代后$X^k, Y^k$的值,发现$\{X^k\}$收敛, 这说明A的按模最小特征值仅有一个且
为正,因此每次取$X^k$中模最大的分量, 即近似于$A^{-1}$的按模最大特征值(A的按模最小特征值倒数), 同时$Y^k$近似为对应特征向量。


\end{document}